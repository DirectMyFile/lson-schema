\documentclass[12pt]{article}
\usepackage{listings}
\usepackage{hyperref}
\hypersetup{
  colorlinks,
  citecolor=black,
  filecolor=black,
  linkcolor=black,
  urlcolor=black
}
\renewcommand{\contentsname}{Table of Contents}
\renewcommand{\thepage}{\roman{page}}
\renewcommand{\thepage}{\arabic{page}}

\begin{document}

\title{LSON Schema}
\author{Logan Gorence \and Kenneth Endfinger}
\date{February 2015}
\maketitle
\pagenumbering{Roman}
\newpage

\tableofcontents
\newpage

\pagenumbering{arabic}
\section{Introduction}
LSON(Laxed JSON) is a heavily modified subset of JSON. It includes a fair amount of changes that allow JSON to be used in a much  broader environment.
\section{Examples}
\subsection{Variations in keywords}
\begin{lstlisting}
{
  /* Yes and no are first-class citizens in boolean values */
  enabled: true,
  confirm: yes,
  /* Boolean values are also case-insensitive */
  property: YES,
  property: Yes
}
\end{lstlisting}
\subsection{Comments}
\begin{lstlisting}
{
  // Single-line comments
  /*
    Multi-line comments
  */
  # Bash style comments
}
\end{lstlisting}
\subsection{Property keys without quotes}
\begin{lstlisting}
{
  /* Properties are not required to have quotes around the keys */
  key: "value"
}
\end{lstlisting}
\end{document}