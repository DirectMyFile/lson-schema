\documentclass[12pt]{article}
\usepackage{listings}
\usepackage{hyperref}
\hypersetup{
  colorlinks,
  citecolor=black,
  filecolor=black,
  linkcolor=blue,
  urlcolor=black
}
\renewcommand{\contentsname}{Table of Contents}
\renewcommand{\thepage}{\roman{page}}
\renewcommand{\thepage}{\arabic{page}}

\begin{document}

\title{LSON Schema}
\author{Logan Gorence \and Kenneth Endfinger}
\date{February 2015}
\maketitle
\pagenumbering{Roman}
\newpage

\tableofcontents
\newpage

\pagenumbering{arabic}
\section{Introduction}
LSON(Laxed JSON) is a heavily modified subset of JSON. It includes a fair amount of changes that allow JSON to be used in a much  broader environment.

\section{Examples}
\subsection{Boolean primitive values}
\label{booleanPrimValueExample}
\begin{lstlisting}
{
  // Yes and no are first-class citizens in boolean values
  enabled: true,
  confirm: yes,
  // Boolean values are case-insensitive
  property: YES,
  property: Yes
}
\end{lstlisting}

\subsection{Comments}
\label{commentsExample}
\begin{lstlisting}
{
  // Single-line comments
  /*
    Multi-line comments
  */
  # Bash style comments
}
\end{lstlisting}

\subsection{String quotes}
\label{stringQuotesExample}
\begin{lstlisting}
{
  /* Properties are not required to have quotes around the keys */
  key: "value"
}
\end{lstlisting}

\subsection{Hexadecimal}
\label{hexExample}
\begin{lstlisting}
{
  // Hexadecimal value
  hex: 0xdeadbeef
}
\end{lstlisting}

\section{Features}
\subsection{Booleans}
Booleans can also use yes/no primitive values that act as true/false values.

See example: \ref{booleanPrimValueExample}

\subsection{Comments}
Multiple types of comments are featured in LSON. Single-line comments, multi-line comments, and bash-style comments are all supported.

See example: \ref{commentsExample}

\subsection{Strings}
String keys and values are quote-optional.

See example: \ref{stringQuotesExample}

\subsection{Hexadecimal}
Hexadecimal values are supported

See example: \ref{hexExample}

\end{document}